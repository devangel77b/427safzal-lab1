\documentclass[reprint,amsmath,amssymb,aps,twoside]{revtex4-2}

\usepackage{graphicx}
\usepackage{amsmath,amssymb,amsfonts}
\usepackage{dcolumn}
\usepackage{bm}
\usepackage{siunitx}
%\usepackage{tikz,pgfplots}
\sisetup{separate-uncertainty=true,multi-part-units=single}
\usepackage[colorlinks,allcolors=blue]{hyperref}
\usepackage{cleveref}
\crefname{equation}{}{}
\crefname{figure}{Fig.}{Figs.}
\crefname{table}{Table}{Tables}
\usepackage{svg}
\svgpath{{./figures}}

% set PDF metadata
\hypersetup{%
pdftitle={Investigating how air resistance influences the motion of a penny and a feather},
pdfauthor={Sarah Afzal, Shreena Desai, and Ansh Modani},
}
\usepackage{fancyhdr}
\pagestyle{fancy}
\fancyhf{}
\fancyhead[RE,RO]{J S\&E \textbf{2}, 57--60 (2026)}
\fancyhead[LO]{Afzal \emph{et al}}
\fancyhead[LE]{How air resistance influences a penny and a feather}
\fancyfoot[C]{\thepage}
\fancypagestyle{mytitlepage}{
\fancyhf{}
\fancyhead[C]{Journal of Science \& Engineering \textbf{2}, 57--60 (2026)}
\fancyfoot[C]{\thepage}
}


\begin{document}
\setcounter{page}{57}

\title{Investigating how air resistance influences the motion of a penny and a feather}
\author{Sarah Afzal}
\email{Contact author: 427safzal@frhsd.com}
\author{Shreena Desai}
\author{Ansh Modani}
\affiliation{Science \& Engineering Magnet Program, \href{https://manalapan.frhsd.com/}{Manalapan High School}, Englishtown, NJ 07726 USA}
\date{\today}

\begin{abstract}
This experiment aims to investigate the influence of air resistance on falling objects with different masses by comparing the motion at which a penny and a feather fall in environments with and without air. Creating a vacuum seal in the tube and using a high-speed video recorder, the position of each object was observed over time for five trials in each environment. In our air-filled environment, the feather fell significantly slower than the penny, almost floating downwards. On the other hand, in vacuum conditions, both objects exhibited similar position-time behavior. These results are consistent with air resistance significantly affecting the motion of low-mass, high-area objects such as feathers. In an environment without air resistance, objects experience a constant acceleration, resulting in increasing velocity over time. 
\end{abstract}

\keywords{keywords here}

\maketitle\thispagestyle{mytitlepage}




\section{Introduction}
This experiment investigates how air resistance affects the vertical motion of a feather and a penny in a controlled tube under two conditions: with air present and with air partially removed. The motion was observed under two different conditions: We used a tube that contained air, and then the same tube, except with the air removed through a vacuum. In an idealized system without air resistance, all objects near Earth’s surface accelerate at the same rate due to gravity. However, in everyday conditions, the presence of air resistance acts against the acceleration due to gravity (\qty{9.8}{\meter\per\second\squared}) and reduces the net acceleration of lighter objects, particularly the feather \cite{tipler,openstax,barrons}. By observing the feather and penny in conditions with and without air, our experiment demonstrates the influence of air resistance on objects in free fall. It provides evidence of their effects on each object’s speed. We hypothesized that if objects fall in an environment with air resistance, then they will not experience constant acceleration, and may approach terminal velocity due to the opposing force of air resistance slowing down the net acceleration. However, if objects fall in an environment without air resistance, then they will accelerate uniformly due to gravity, resulting in similar accelerations due to gravity.





\section{Methods and materials}
We conducted this experiment using a meter stick, a few pieces of tape, a feather weighing \qty{0.025}{\gram}, a penny weighing \qty{0.050}{\gram}, a transparent acrylic tube (inner diameter \qty{8.00}{\centi\meter}; Flinn Scientific; Batavia, IL), connected to a laboratory vacuum pump (JB Industries DV-85N; Aurora, IL) via a valve system to reduce air density inside the tube. 

The resulting motion of objects was recorded  with an smartphone (iPhone 14; Apple, Inc; Cupertino, CA) operated at \qty{240}{frame\per\second}. The camera was mounted stationary and perpendicular to the tube to minimize perspective distortion. The high fps ensured that the objects in fall would be captured throughout, reducing human reaction error and allowing us to collect more accurate position data. To minimize other data collection errors from the video and because the tube had to be flipped, causing unsteadiness, we taped the meter stick to the tube to provide a consistent scale reference. This allowed us not to have to limit our range of motion and could flip the tube easily, all while still being able to capture a measurement of position for each of the objects in the video. To obtain the positions of each object in the presence of air resistance, the tube lid was sealed, containing both the penny and the feather, and an open valve that allowed air inside of it. Before we started to film, the tube had already been flipped upside down, and we filmed and used the data of it being flipped right side up for a total of five trials in order to average out the positions captured with the aim to minimize confounding variables that one trial may have. 

To create the air-resistance-free environment, we evacuated the tube using the vacuum pump, which took around ten minutes, and then isolated the valve using two valves, one on the tube and one on the vacuum pump. Data collection proceeded as previously, for a total of five trials. 

Video kinematics were digitized manually. The position of the penny and feather were recorded every \qty{0.25}{\second} for each trial of each object. Velocity was calculated using:
\begin{equation}
V_{avg} = \left| \frac{\Delta y}{\Delta t} \right| 
\end{equation}
To associate each average velocity with a time value, a midpoint time was calculated by finding the midpoint between two times, or by using the following equation:
\begin{equation}
t_{midpoint} = \dfrac{t_{next} - t_{current}}{2}
\end{equation}

Statistical analyses \cite{starnes:2015:practice} were conducted in R \cite{R} using the \texttt{dplyr} and \texttt{ggplot2} libraries \cite{dplyr,ggplot2}. Data and code are available at \url{https://github.com/devangel77b/427safzal-lab1}.




\section{Results}
Position data for the feather and penny with air resistance present are shown in \cref{fig:1} and \cref{tab:1,tab:2}. For the feather, the slope is $v=\qty{-0.53\pm0.06}{\meter\per\second}$ (linear regression, $R^2=0.92$, $p=\num{8.49e-5}$). For the penny, the slope is $v=\qty{-2.00\pm0.07}{\meter\per\second}$ (linear regression, $R^2=0.997$, $p=0.02$).
\begin{figure}
\begin{center}
\includesvg{fig1.svg}
\end{center}
\caption{Position versus time with air resistance present. For the feather, the slope is $v=\qty{-0.53\pm0.06}{\meter\per\second}$ (linear regression, $R^2=0.92$, $p=\num{8.49e-5}$). For the penny, the slope is $v=\qty{-2.00\pm0.07}{\meter\per\second}$ (linear regression, $R^2=0.997$, $p=0.02$).}
\label{fig:1}
\end{figure}

Position data without air resistance are shown in \cref{fig:2} and \cref{tab:3,tab:4}. There are not significant diffences between penny and feather (ANOVA, $p=0.1388$); for the pooled data for both penny and feather, $v=\qty{-0.808\pm0.007}{\meter\per\second}$ (linear regression, $R^2=0.9992$, $p<\num{2e-16}$.
\begin{figure}
\begin{center}
\includesvg{fig2.svg}
\end{center}
\caption{Position versus time with air resistance absent. There are not significant diffences between penny and feather (ANOVA, $p=0.1388$); for the pooled data for both penny and feather, $v=\qty{-0.808\pm0.007}{\meter\per\second}$ (linear regression, $R^2=0.9992$, $p<\num{2e-16}$.}
\label{fig:2}
\end{figure}

\begin{table}
  \caption{Midpoints of each time interval and calculated speeds (\unit{\meter\per\second}) for the feather across five trials, used to determine speed constancy when air resistance is present}
  \label{tab:1}
  \begin{ruledtabular}
    \begin{tabular}{cccccc}
      $t$, \unit{\second} & trial 1 & trial 2 & trial 3 & trial 4 & trial 5 \\
      \colrule
      \num{0.125} & \num{0.396} & \num{0.388} & \num{0.292} & \num{0.390} & \num{0.404} \\
      \num{0.375} & \num{0.344} & \num{0.280} & \num{0.333} & \num{0.330} & \num{0.351} \\
      \num{0.625} & \num{0.150} & \num{0.154} & \num{0.104} & \num{0.112} & \num{0.223} \\
      \num{0.875} & \num{0.544} & \num{0.533} & \num{0.474} & \num{0.465} & \num{0.354} \\
      \num{1.125} & \num{0.216} & \num{0.163} & \num{0.115} & \num{0.211} & \num{0.102} \\
      \num{1.375} & \num{0.320} & \num{0.302} & \num{0.260} & \num{0.314} & \num{0.214} \\
      \num{1.625} & \num{0.444} & \num{0.432} & \num{0.355} & \num{0.396} & \num{0.334} \\
%      \num{1.875} & \num{0.042} & \num{0.048} & \num{0.041} & \num{0.046} & \num{0.044} \\
%      \num{2.125} & \num{0.028} & \num{0.023} & \num{0.026} & \num{0.021} & \num{0.024} \\
%      \num{2.275} & \num{0.040} & \num{0.036} & \num{0.034} & \num{0.039} & \num{0.037} \\
%      \num{2.625} & \num{0.042} & \num{0.045} & \num{0.038} & \num{0.042} & \num{0.040} \\
    \end{tabular}
  \end{ruledtabular}
\end{table}

% the last 4 rows are where the feather is stopped. also why the hell are there so many mistakes in the raw data. 

%\begin{table}
%\caption{Midpoints of each time interval and calculated speeds (\unit{\meter\per\second}) for the feather across five trials, used to determine speed constancy when air resistance is present.}
%\label{tab:1}
%\end{table}

\begin{table}
  \caption{Midpoints of each time interval and calculated speed (\unit{\meter\per\second}) for the penny across five trials, used to determine speed constancy when air resistance is present.}
  \label{tab:2}
  \begin{ruledtabular}
    \begin{tabular}{cccccc}
      $t$, \unit{\second} & trial 1 & trial 2 & trial 3 & trial4 & trial 5 \\
      \colrule
      \num{0.125} & \num{0.214} & \num{0.214} & \num{0.214} & \num{0.213} & \num{0.214} \\
      \num{0.375} & \num{0.186} & \num{0.187} & \num{0.186} & \num{0.186} & \num{0.186} \\
      % rest of entries are zero as penny has hit end
    \end{tabular}
  \end{ruledtabular}
\end{table}

% this table is a waste of space. why are you filling the page with zeroes.
% you do not get more points for having more pages.  

%\begin{table}
%\caption{Midpoints of each time interval and calculated speeds (\unit{\meter\per\second}) for the penny across five trials, used to determine speed constancy when air resistance is present.}
%\label{tab:2}
%\end{table}

\begin{table}
  \caption{Midpoints of each time interval and calculated speeds (\unit{\meter\per\second}) used to determine constancy in the absence of air resistance for the feather}
  \label{tab:3}
  \begin{ruledtabular}
    \begin{tabular}{cccccc}
      $t$, \unit{\second} & trial 1 & trial 2 & trial 3 & trial 4 & trial 5 \\
      \colrule
      \num{0.125} & \num{0.781} & \num{0.780} & \num{0.780} & \num{0.780} & \num{0.780} \\
      \num{0.375} & \num{0.785} & \num{0.785} & \num{0.785} & \num{0.785} & \num{0.779} \\
      \num{0.625} & \num{0.800} & \num{0.800} & \num{0.801} & \num{0.800} & \num{0.782} \\
      \num{0.875} & \num{0.805} & \num{0.805} & \num{0.805} & \num{0.806} & \num{0.800} \\
      \num{1.125} & \num{0.800} & \num{0.800} & \num{0.800} & \num{0.802} & \num{0.800} \\
    \end{tabular}
  \end{ruledtabular}
\end{table}

%\begin{table}
%\caption{Midpoints of each time interval and calculated speeds (\unit{\meter\per\second}) for the feather across five trials, used to determine speed constancy in the absence of air resistance for the feather.}
%\label{tab:3}
%\end{table}

\begin{table}
  \caption{Midpoints of each time interval and calculated speeds (\unit{\meter\per\second}) used to determine constancy in the absence of air resistance for the penny}
  \label{tab:4}
  \begin{ruledtabular}
    \begin{tabular}{cccccc}
      $t$, \unit{\second} & trial 1 & trial 2 & trial 3 & trial 4 & trial 5 \\
      \num{0.125} & \num{0.800} & \num{0.800} & \num{0.801} & \num{0.800} & \num{0.800} \\
      \num{0.375} & \num{0.811} & \num{0.812} & \num{0.811} & \num{0.811} & \num{0.811} \\
      \num{0.625} & \num{0.812} & \num{0.811} & \num{0.812} & \num{0.811} & \num{0.811} \\
      \num{0.875} & \num{0.800} & \num{0.801} & \num{0.800} & \num{0.801} & \num{0.800} \\
      \num{1.125} & \num{0.801} & \num{0.801} & \num{0.801} & \num{0.801} & \num{0.801} \\
    \end{tabular}
  \end{ruledtabular}
\end{table}




\section{Discussion}
\subsection{With air resistance, the penny and feather fall at different velocities}
Average velocities were computed over fixed intervals; this does not imply constant velocity or that terminal velocity was reached. Objects falling in air may reach a constant velocity after sufficient time due to terminal velocity; the finite tube length likely limited the extent to which full terminal velocity was reached in this experiment. Not only that, but they exhibited similar accelerations, both hitting the ground at roughly the same time. \Cref{fig:1} displays the position of the feather and the penny in relation to time, both color-coded. The feather initially accelerated and then approached a regime where drag significantly reduced further acceleration where drag forced a significant reduction in further acceleration, signalling that the amount of air resistance was outweighing the feather itself. On the contrary, the penny had fallen much quicker. The amount of air resistance that the object is affected by can be determined by two things: the speed of the object and the cross-sectional area of the object, or the relative magnitude of drag forces compared to gravitational force \cite{cioci:2022:galileo}. This is exemplified through the feather, as it is tiny, weighs much less than a penny, and has soft, ridge-like corners that catch even more air, allowing for more air resistance. The penny, which is made of copper, making it more dense and heavier, does not allow for much air resistance, allowing it to fall and hit the ground almost immediately. The changing velocity is shown once again in \cref{tab:1}, where it is suggested that the speeds of the feather and the penny are not the same or even similar. It is, however, surprising how high the $R^2$ of the feather in \cref{fig:1} is, especially since the speeds are nowhere near being the same. Either way, a linear position-time relationship would indicate constant velocity; however, accelerating motion is expected prior to reaching terminal velocity, neither did the table of speeds show that it is constant in environments involving air resistance.

\subsection{Without air resistance, the penny and feather fall at similar velocities}
In \cref{fig:2}, the plot of the feather and the penny is not only linear to the eye, but also very close to each other in proximity, representing similar position-time behavior consistent with constant acceleration and the same rate within each object. Supporting its linearity, the $R^2$ values of both the feather and the penny are 0.963 and 0.982, respectively, showing great accountability by the LSRL for all values in the data set. As seen in \cref{tab:2}, the speeds of both objects, after being calculated, are in fact similar, and it would be reasonable to conclude that both objects experienced similar accelerations due to gravity just as should happen in an air-resistance-free environment.

\subsection{Sources of experimental error}
Several possible sources of systemic error may have contributed to vouching for or straying away from a constant speed in an air-resistance-free environment by around 5\%-10\%. For example, the diameter of our tube was quite small, possibly conflicting with the actual position of each object as they collided into the wall and each other. This is especially the case for the feather and may account for its high $R^2$ value as given in \cref{fig:1}. As the feather hits the sides of the tube, it may seem like it is falling more slowly when it really should not. If the penny had interacted with the feather at all, it could have pushed it down with it, causing the rate to be the same for both objects in the air-resistance-free environment when it may not have been.

Possible systematic errors include friction between the objects and the tube walls, interaction between the feather and penny, and uncertainties introduced during the tube inversion. These effects may influence the measured motion and limit generalization beyond this setup.





\section{Acknowledgements} 
We thank several anonymous reviewers for providing helpful comments. SA produced the hypothesis, collected the data, and graphed it, allowing us to accurately depict the trends on the graph. AM was in charge of flipping the tube, ensuring that the feather and coin hit the side of the tube as little as possible to minimize outside factors that might interfere with the experiment and create a more accurate representation of no air resistance. SD was responsible for recording the experiment, ensuring the frame rate was accurate and that the data could be made intelligible from the video.






\bibliography{lab.bib}
%References 

%Vioci, Vicenzo. 2022. Galileo's falling bodies into the history of physics and the nature of science as a case study. Education. Universite de Lille. English


\end{document}
